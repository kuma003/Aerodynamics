\documentclass[uplatex,dvipdfmx,a4j,11pt]{jsarticle}

\usepackage[utf8]{inputenc}
\usepackage{amsmath,amsfonts}
\usepackage{bm}
\usepackage{amssymb}
\usepackage{otf}
\usepackage{pxrubrica}
\usepackage{ascmac}
\usepackage{ar}
% \usepackage{wrapfig}
\usepackage{here}
\usepackage[hang,small,bf]{caption}
\usepackage[subrefformat=parens]{subcaption}
\captionsetup{compatibility=false}

\usepackage{graphicx}
\usepackage{comment}
\usepackage{color}
\usepackage{url}
\usepackage{siunitx}
\usepackage[version=4]{mhchem}
\usepackage{paralist}
\usepackage{longtable}
\usepackage{multirow}
\usepackage[dvipdfmx]{hyperref}
\usepackage{pxjahyper}
\usepackage{cleveref}
\crefname{figure}{図}{図}
\crefname{equation}{式}{式}
\crefname{table}{表}{表}
\crefname{section}{節}{節}
\newcommand{\crefpairconjunction}{と}
\newcommand{\crefrangeconjunction}{から}
\newcommand{\crefmiddleconjunction}{,}
\newcommand{\creflastconjunction}{, および}

\usepackage{fancyhdr}
% \usepackage{lastpage}
\fancypagestyle{mypagestyle}{%
\lhead{}%ヘッダ左を空に
\rhead{}%ヘッダ右を空に
% \cfoot{\thepage/\pageref*{LastPage}}%フッタ中央に"今のページ数/総ページ数"を設定
\renewcommand{\headrulewidth}{0.0pt}%ヘッダの線を消す
}
\pagestyle{mypagestyle}

% コマンド定義
\newcommand{\divergence}{\mathrm{div}\,}  %ダイバージェンス
\newcommand{\grad}{\mathrm{grad}\,}  %グラディエント
\newcommand{\rot}{\mathrm{rot}\,}  %ローテーション
\newcommand{\diff}{\mathrm{d}} % 微分
\newcommand{\e}{\mathbf{e}} % 単位ベクトル



\title{Barrowman Method を用いた解析について}
\date{}
\author{佐藤 空馬 (14期代空力班長)}

\begin{document}
\maketitle
\thispagestyle{mypagestyle}%タイトルページ等ではpagestyleが変更されるので改めて設定する

\section{はじめに}
Barrowman Method\cite{Barrowman}は,ロケットの空力解析を簡便に行うための解析的・経験的手法である.
また,同様の手法としてはOpenRocketのドキュメント\cite{OpenRocket}も詳しく,Barrowman Methodを中心に据えて種々の補正を取り入れている.
本稿は,両者を中心にロケットの空力諸元の導出をまとめるものである.

以下,式番号において,Barrowman\cite{Barrowman}と同一のものは(B.0.0),OpenRocket\cite{OpenRocket}と同一のものは(OR.0.0)と表記する.

\subsection{Barrowman Method}
Barrowman Methodにおいては,以下の仮定がなされていることに注意が必要である:
\begin{itemize}
  \item 迎角は非常に小さい
  \item 流れは定常 (steady) かつ渦なし (irrotational) (つまりポテンシャル流 (potential flow))
  \item 機体は剛体
  \item ノーズコーンの先端は尖っている
\end{itemize}
OpenRocketでは特にモデルロケットのようなケースを想定しているため,落下時のような迎角が大きい場合についても言及されているが,
本稿ではこの仮定の下で議論を進める.

\section{ノーズコーン/ボディ/テールの法線力・圧力中心}
ノーズコーンやボディ,テールは軸対称であるため本質的には同様に扱うことができる.

ここでは議論が比較的見通しやすいBarrowmanの方を中心に説明する.

\subsection{亜音速において}
細長く軸対称な物体においては,単位長さあたり揚力は以下のように表される:
\begin{equation}
  n(x) = \rho V \frac{\partial}{\partial x} \left[A(x)w(x)\right]
  \tag{B.3.57}
  \label{eq:barrowman_lift}
\end{equation}
ここで,$A(x)$は断面積,$w(x)$は機軸に対して垂直な方向の速度成分で,位置$x$には依存せず,
\begin{equation*}
  w(x) = V \sin\alpha \approx V\alpha.
\end{equation*}
したがって,
\begin{equation}
  n(x) = \rho V^2 \alpha \frac{\diff A(x)}{\diff x}
  \tag{B.3.59}
\end{equation} 
となる.
これからただちに径が変化しない\textbf{ボディ部にはたらく揚力は0となる}ことが分かる.

機体径は滑らかに変化するという仮定の下に,これを全体にわたって積分すると法線力$N$は\footnote{ここからBarrowmanの議論とは少し異なるが本質的に同じである.},
\begin{equation}
  N = \int_0^{l } n(x) \,\diff x =  \rho V^2 \alpha [A(l) - A(0)]
\end{equation}
となる (ここで$l$は対象とするコンポーネント長さ).
したがって,法線力係数は,
\begin{equation}
  C_{N} = \frac{N}{\frac{1}{2}\rho V^2 A_\mathrm{ref}} = 2\frac{A(l) - A(0)}{A_\mathrm{ref}} \alpha
  % \tag{3--60\cite{Barrowman}}
\end{equation}
例えば,ノーズコーンなら,$A(0) = 0, A(l) = A_\mathrm{ref}$より,
\begin{equation}
  C_{N, \mathrm{nose}} = 2\alpha\footnotemark
\end{equation}
\footnotetext{なお,この結果は複素ポテンシャルによって計算された玉木\cite{tamaki} 式$(5\cdot 6)$の結果とも一致する.}
テールコーンなら,$A(0) = A_\mathrm{ref}, A(l) = 0$より,
\begin{equation}
  C_{N, \mathrm{tail}} = - 2 \left(1 - \frac{A(l)}{A_\mathrm{ref}}\right) \alpha
\end{equation}
となる.

法線力係数傾斜についてはこれより,
\begin{equation}
  C_{N\alpha} = \left.\frac{\partial C_{N}}{\partial \alpha}\right|_{\alpha \to 0} = 2\frac{A(l) - A(0)}{A_\mathrm{ref}}
  \tag{B.3.65}
\end{equation}
つまり,ノーズ法線力係数傾斜は,
\begin{equation}
  C_{N\alpha, \mathrm{nose}}  = 2
\end{equation}
テール法線力係数傾斜については,
\begin{equation}
  C_{N\alpha, \mathrm{tail}} = - 2 \left(1 - \frac{A(l)}{A_\mathrm{ref}}\right)
\end{equation}
となる.
ここで,このことから,\textbf{テールが生み出す法線力は負である}ことがわかる.

また,圧力中心位置については,
\begin{equation}
  X_\mathrm{cp} = \frac{\int_0^{l } x n(x) \,\diff x}{\int_0^{l } n(x) \,\diff x}
  = \frac{[x A(x)]_0^{l } - \int_0^{l } A(x) \,\diff x}{[A(x)]_0^{l }}
  = \frac{l  A(l ) - \mathrm{Vol}}{A(l ) - A(0)}
  \tag{OP.3.28}
\end{equation}
ここで,$\mathrm{Vol} = \int_0^{l } A(x)\,\diff x$はコンポーネント体積であって,
ここでは式の形をOpenRocketに合わせている.

つまり,このことから機体全長に対するノーズコーンの圧力中心位置は,
\begin{equation}
  \frac{X_{cp, \mathrm{nose}}}{l } = 1 - \frac{\mathrm{Vol}}{A(l ) l }\footnotemark
\end{equation}
\footnotetext{この表式は玉木\ref{tamaki}式$(5\cdot 7)$に合わせた.}
のように表される(テールコーンについてはほぼ同じなので省略).

\subsubsection{超音速において}

工事中.

いつかかく.

\subsection{フィンの法線力・圧力中心}

ここでは,OpenRocketを参考に説明する.

\enskip

1枚のフィンが生み出す法線力係数傾斜はDiederichの半経験的式によって与えられる:
\begin{equation}
  (C_{N_\alpha})_1 = \frac{C_{N_{\alpha0}}F_\mathrm{D} (\frac{A_\mathrm{fin}}{A_\mathrm{ref}}\cos\Gamma_\mathrm{c})}{2 + F_\mathrm{D} \sqrt{1 + \frac{4}{F_\mathrm{D}^2}}}
  \tag{OR.3.37}
\end{equation}
ここで,
\begin{itemize}
  \item $C_{N_{\alpha0}}$: 2次元平板の法線力係数
  \item $F_\mathrm{D}$: Diederichの形状補正係数
  \item $A_\mathrm{fin}$: フィン面積
  \item $\Gamma_\mathrm{c}$: 中点の角度 (\cref{fig:fin_geometry}参照)
\end{itemize}
\begin{figure}[htbp]
  \centering
  \includegraphics[width=0.4\linewidth]{Barrowman/img/fin_geometry.pdf}
  \caption{フィンの幾何学的パラメータ.}
  \label{fig:fin_geometry}
\end{figure}

ポテンシャル流から求められる2次元平板の法線力係数$2\pi$に,圧縮性流体に対する補正: Prandtl-Glauert補正を施して,
\begin{equation}
  C_{N_{\alpha0}} = \frac{2\pi}{\sqrt{1 - M^2}} = \frac{2\pi}{\beta}
  \tag{OR.3.38}
\end{equation}
である.

また,Diederichの形状補正係数$F_\mathrm{D}$は,次で与えられる:
\begin{equation}
  F_\mathrm{D} = \frac{\AR}{\frac{1}{2}C_{N_{\alpha0}}\cos\Gamma_\mathrm{c}}
  \tag{OR.3.39}
\end{equation}
ここで,$\AR$はフィンのアスペクト比であって,$\AR \equiv 2s^2/A_\mathrm{fin}$である.

以上の式を用いて整理すると,
\begin{equation}
  (C_{N_\alpha})_1 = \frac{2\pi \frac{s^2}{A_\mathrm{ref}}}{1 + \sqrt{1 + \left(\frac{\beta s^2}{A_\mathrm{fin}\cos\Gamma_\mathrm{c}}\right)^2}}
\end{equation}
\end{document}
